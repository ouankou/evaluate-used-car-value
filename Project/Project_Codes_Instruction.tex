%\documentclass{article}


\documentclass[english,11pt]{article}
\usepackage{fullpage}
\usepackage[latin1]{inputenc}
\usepackage[T1]{fontenc}
\usepackage{lmodern}
\setlength{\parskip}{\medskipamount}
\setlength{\parindent}{0pt}



\usepackage{amsmath}
\usepackage{amssymb}
\usepackage{amsthm}
\usepackage{amsfonts}
\usepackage{graphicx}
\usepackage{mathtools}
\usepackage{hyperref}
\usepackage{xspace}
\usepackage{aliascnt}

\newcommand{\lfalse}{\textbf{false}\xspace}
\newcommand{\ltrue}{\textbf{true}\xspace}
\newcommand{\Int}{\mathbb{Z}}
\newcommand{\IntMod}[1]{\Int_{#1}}
\newcommand{\unique}{\exists!}
\renewcommand{\gets}{\mathrel{\mathop:}=}
\newcommand{\nameq}{\stackrel{\textup{\tiny def}}{=}}
\newcommand{\acteq}{\longrightarrow}
\newcommand{\actto}{\longmapsto}
\renewcommand{\mod}{{}\textnormal{\texttt{\;mod\;}}{}}
\newcommand{\modop}[1]{\textnormal{\texttt{\;mod\;}}#1}

\newcommand{\XXMacroMagic}{}
\newcommand{\alright}[1]{&\quad&\text{#1}}

\newcommand{\textiff}{\textit{iff}\xspace}
\newcommand{\quicksec}[1]{\noindent{\bf #1.}\;}
\newcommand{\funcname}[1]{\textsc{#1}}
\newcommand{\funcall}[2]{\funcname{#1}(#2)}
\newcommand{\vbl}[1]{\textnormal{\textit{#1}}}
\newcommand{\ttvbl}[1]{\texttt{#1}}
\newcommand{\ttaref}[2]{\ttvbl{#1[}#2\ttvbl{]}}
\newcommand{\ttmemloc}[1]{\texttt{\&#1}}
\newcommand{\mathmemloc}[1]{\ensuremath{\texttt{\&}#1}}

\DeclarePairedDelimiter{\abs}{\lvert}{\rvert}
\DeclarePairedDelimiter{\ceil}{\lceil}{\rceil}
\DeclarePairedDelimiter{\floor}{\lfloor}{\rfloor}
\DeclarePairedDelimiter{\avect}{\langle}{\rangle}
\DeclarePairedDelimiter{\aset}{\allowbreak\lbrace}{\rbrace}

\newcommand{\lxor}{\veebar}
\newcommand{\true}{\top}
\newcommand{\fals}{\bot}
\newcommand{\ket}[1]{\lvert #1\rangle}
%\newcommand{\Nat}{\mathbb{N}}
\newcommand{\WildSym}{\texttt{X}\xspace}
\newcommand{\transpose}{^ \top }
\newcommand{\queseq}{\overset{?}{=}}
%\newcommand{\transpose}{^ \intercal }
\newcommand{\pipe}{\,|\,}
\newcommand{\detop}[1]{\det(#1)}
%\newcommand{\vect}[1]{\boldsymbol{\vbl{#1}}}
\newcommand{\vect}[1]{{\overrightarrow{#1}}}
%\newcommand{\slfrac}[2]{\left.#1\middle/#2\right.}
\newcommand{\slfrac}[2]{#1/#2}
%\newcommand{\vect}[1]{{\vv{#1}}}
\newcommand{\fmlang}[1]{\textnormal{\textrm{#1}}}
\providecommand{\expten}[1]{\ensuremath{\times 10^{#1}}}

\newenvironment{smalldisplay}%
{\begin{center}\scriptsize}%
{\end{center}}

\def\imagetop#1{\vtop{\null\hbox{#1}}}
\newcommand{\TODO}[1]{\noindent\textcolor{red}{{\bf TODO:} #1}}
\newcommand{\NOTE}[1]{\noindent\textcolor{red}{{\bf NOTE:} #1}}
\newcommand{\QUES}[1]{\noindent\textcolor{red}{{\bf ???:} #1}}

\mathchardef\breakingcomma\mathcode`\,
{\catcode`,=\active
 \gdef,{\breakingcomma\discretionary{}{}{}}
}
\newcommand{\mathlist}[1]{\mathcode`\,=\string"8000 #1}

\newcommand{\quickcol}[2]{%
\begin{tabular}{@{}#1@{}}#2\end{tabular}}

\makeatletter
\@ifclassloaded{beamer}{
%%%%%%% BEG In beamer
%\usetheme{Warsaw}
%\setbeamertemplate{headline}{}
\setbeamertemplate{bibliography item}[text]
\setbeamertemplate{navigation symbols}{%
 \begin{tabular}{r}%
 \insertframenavigationsymbol \\
 {\scriptsize \color{gray} \insertframenumber{}/\inserttotalframenumber{}}
 \end{tabular}%
}
%%%%%%% END In beamer
}{
%%%%%%% BEG Not in beamer
\usepackage{paralist}

\newenvironment{itemize*}%
{\begin{compactitem}}%
{\end{compactitem}}

\newenvironment{enumerate*}%
{\begin{compactenum}}%
{\end{compactenum}}

\@ifundefined{ifusesection}{%
 \newif\ifusesection %
 \usesectiontrue %
}{}

\ifusesection
 \newtheorem{theorem}{Theorem}[section]
\else
 \newtheorem{theorem}{Theorem}
\fi

\def\sectionautorefname{Section}
%\def\subsectionautorefname{Subsection}
\def\subsectionautorefname{Section}
\def\subsubsectionautorefname{Section}
\def\chapterautorefname{Chapter}

% counter for numbering, and make them work with \autoref.
\newcommand{\mynewtheorem}[2]{
 \newaliascnt{#1}{theorem}
 \newtheorem{#1}[#1]{#2}
 \aliascntresetthe{#1}
 % maybe we will squish some autoref defaults, but who cares?
 \expandafter\def\csname #1autorefname\endcsname{#2}
}
\mynewtheorem{lemma}{Lemma}
\mynewtheorem{corollary}{Corollary}
\mynewtheorem{example}{Example}
\theoremstyle{definition}
\mynewtheorem{problem}{Problem}
\mynewtheorem{definition}{Definition}
%\theoremstyle{remark}
\mynewtheorem{remark}{Remark}
\mynewtheorem{observation}{Observation}

\def\ALC@uniqueautorefname{Line}

%%% hyperref link stuff
\hypersetup{
 pdfborderstyle={/S/U/W 0.5}
}
%%%%%%% END Not in beamer
}
\makeatother

\newcounter{exercise}
\def\theexercise{\arabic{exercise}}
\newenvironment{exercise}[1][]%
{\allowbreak\begin{samepage}\vspace*{2em}\hrule%
 \refstepcounter{exercise}
 {\bf Exercise \theexercise.}\ifx\newenvironment#1\newenvironment\else\space(#1)\fi\\}
{\vspace*{0.5em}\hrule\end{samepage}\vspace*{1em}}


\newcounter{exercisepart}[exercise]
\def\theexercisepart{\theexercise.\alph{exercisepart}}
\newenvironment{exercisepart}[1][]%
{\allowbreak\begin{samepage}\refstepcounter{exercisepart}
 {\bf Exercise \theexercisepart.}\ifx\newenvironment#1\newenvironment\else\space(#1)\fi\space}
{\end{samepage}

}

\def\exerciseautorefname{Exercise}
\def\exercisepartautorefname{Exercise}

\usepackage{algorithm}
\usepackage{algorithmic}

\providecommand\algorithmname{Algorithm}
\providecommand{\subfigureautorefname}{\figureautorefname}

\newcommand{\algorithmicoutput}{\textbf{Output:}}
\newcommand{\OUTPUT}{\item[\algorithmicoutput]}

\newcommand{\LET}{\STATE \textbf{let}\xspace}



\usepackage[english]{babel}

\renewcommand{\labelenumii}{\arabic{enumi}.\arabic{enumii}.}
\renewcommand{\labelenumiii}{\arabic{enumi}.\arabic{enumii}.\arabic{enumiii}.}
\renewcommand{\labelenumiv}{\arabic{enumi}.\arabic{enumii}.\arabic{enumiii}.\arabic{enumiv}.}

\usepackage{tikz}
\usetikzlibrary{arrows}
\usetikzlibrary{automata}
\usetikzlibrary{shapes}
\usetikzlibrary{backgrounds}

\newenvironment{solution}{}{}



\usepackage{pgf}
\usepackage{tikz}
\usetikzlibrary{arrows,automata}
\usepackage{amsmath}
\usepackage{verbatim}
\usepackage{float}
\usepackage{tikz-qtree}
\usepackage{pdflscape}
\usetikzlibrary{shapes.multipart}
\usepackage{booktabs}

\usepackage{graphicx} % insert image
\usepackage{layout}
\usepackage[procnames]{listings}
\usepackage{color}
\usepackage[margin=1.5in]{geometry}
\usepackage{tcolorbox}

\tcbuselibrary{listings,skins,breakable}
\newtcblisting{mycode}{
      arc=0mm,
      top=0mm,
      bottom=0mm,
      left=3mm,
      right=0mm,
      width=\textwidth,
      boxrule=1pt,
      listing only,
      listing options={},
      breakable
}


\begin{document}

\definecolor{keywords}{RGB}{255,0,90}
\definecolor{comments}{RGB}{0,0,113}
\definecolor{red}{RGB}{160,0,0}
\definecolor{green}{RGB}{0,150,0}
 
\lstset{language=Matlab,
        breaklines=true
        basicstyle=\ttfamily\small, 
        keywordstyle=\color{keywords},
        commentstyle=\color{comments},
        stringstyle=\color{red},
        showstringspaces=false,
        identifierstyle=\color{green},
        procnamekeys={def,class}}



\title{CS5811 Project Source Codes Instruction}
\date{}
\author{Xueling Li (xuelingl@mtu.edu) \\ Kyle Oswald (kloswald@mtu.edu) \\ Anjia Wang (anjiaw@mtu.edu)}
\author{Anjia Wang (anjiaw@mtu.edu)}

\maketitle


\section{Naive Bayesian Network in Python 3}

\definecolor{keywords}{RGB}{255,0,90}
\definecolor{comments}{RGB}{0,0,113}
\definecolor{red}{RGB}{160,0,0}
\definecolor{green}{RGB}{0,150,0}
 
\lstset{language=Python,
        breaklines=true
        basicstyle=\ttfamily\small, 
        keywordstyle=\color{keywords},
        commentstyle=\color{comments},
        stringstyle=\color{red},
        showstringspaces=false,
        identifierstyle=\color{green},
        procnamekeys={def,class}}


Put Python code file with dataset file "car.data" in the same folder and execute the codes in Python 3 environment by "python3 Used\_Car\_Naive\_Bayes\_Python.py".
It will run several times (set by $cv$ variable) and output corresponding accuracy.
Then an average accuracy will be calculated.
It is written by Anjia Wang and runs on Linux or macOS.

\begin{mycode}
def preprocess(data, testsize): # testsize is % of testing set
    df = pandas.read_csv(data, header=None) # generate training and testing set
    training_set, testing_set = train_test_split(df, test_size = testsize)
    # generate training data and label set
    trainX = training_set.iloc[:, 0:6].values.tolist()
    trainY = training_set.iloc[:, 6].values.tolist()
    # generate testing data and label set    
    testX = testing_set.iloc[:, 0:6].values.tolist()
    testY = testing_set.iloc[:, 6].values.tolist()
    return trainX, trainY, testX, testY
\end{mycode}

\begin{mycode}
def term_prob_matrix(trainX, trainY):
	# calculate the term frequency in varied classes.
    label_values = numpy.zeros((4, 21))
    for i in range(0, len(trainX)):
        if trainY[i] == 'unacc':
            cl = 0
        elif trainY[i] == 'acc':
            cl = 1
        elif trainY[i] == 'good':
            cl = 2
        elif trainY[i] == 'vgood':
            cl = 3
        #check first attribute 'buying'
        if trainX[i][0] == 'vhigh':
            label_values[cl][0] += 1
        elif trainX[i][0] == 'high':
            label_values[cl][1] += 1
        elif trainX[i][0] == 'med':
            label_values[cl][2] += 1
        elif trainX[i][0] == 'low':
            label_values[cl][3] += 1
        #check second attribute 'maint'
        if trainX[i][1] == 'vhigh':
            label_values[cl][4] += 1
        elif trainX[i][1] == 'high':
            label_values[cl][5] += 1
        elif trainX[i][1] == 'med':
            label_values[cl][6] += 1
        elif trainX[i][1] == 'low':
            label_values[cl][7] += 1
        #check third attribute 'doors'
        if trainX[i][2] == '2':
            label_values[cl][8] += 1
        elif trainX[i][2] == '3':
            label_values[cl][9] += 1
        elif trainX[i][2] == '4':
            label_values[cl][10] += 1
        elif trainX[i][2] == '5more':
            label_values[cl][11] += 1
        #check fourth attribute 'persons'
        if trainX[i][3] == '2':
            label_values[cl][12] += 1
        elif trainX[i][3] == '4':
            label_values[cl][13] += 1
        elif trainX[i][3] == 'more':
            label_values[cl][14] += 1
        #check fifth attribute 'lug_boot'
        if trainX[i][4] == 'small':
            label_values[cl][15] += 1
        elif trainX[i][4] == 'med':
            label_values[cl][16] += 1
        elif trainX[i][4] == 'big':
            label_values[cl][17] += 1
        #check sixth attribute 'safety'
        if trainX[i][5] == 'low':
            label_values[cl][18] += 1
        elif trainX[i][5] == 'med':
            label_values[cl][19] += 1
        elif trainX[i][5] == 'high':
            label_values[cl][20] += 1
    delta = 0.06
    i = 0
    for row in label_values:
        label_sum = sum(row)
        j = 0
        for v in row:
            label_values[i][j] = (1-delta)*v/label_sum + delta/21
            j += 1
        i += 1
    return label_values	
\end{mycode}

\begin{mycode}
def calc_prior_prob(trainY):
    # calculate prior probability
    prior_prob = numpy.zeros(4)
    for i in range(0, len(trainY)):
        if trainY[i] == 'unacc':
            prior_prob[0] += 1
        elif trainY[i] == 'acc':
            prior_prob[1] += 1
        elif trainY[i] == 'good':
            prior_prob[2] += 1
        elif trainY[i] == 'vgood':
            prior_prob[3] += 1
    prior_prob = prior_prob/sum(prior_prob)
    return prior_prob
\end{mycode}

\begin{mycode}
def predict(testX, term_matrix, prior_prob):
	# prediction on testing data set.
    test_results = numpy.zeros((len(testX),4))
    for i in range(0,len(testX)):
        for j in range(0,4):
            test_results[i][j] += log(prior_prob[j])
            #check first attribute 'buying'
            if testX[i][0] == 'vhigh':
                test_results[i][j] += log(term_matrix[j][0])
            elif testX[i][0] == 'high':
                test_results[i][j] += log(term_matrix[j][1])
            elif testX[i][0] == 'med':
                test_results[i][j] += log(term_matrix[j][2])
            elif testX[i][0] == 'low':
                test_results[i][j] += log(term_matrix[j][3])
            #check second attribute 'maint'
            if testX[i][1] == 'vhigh':
                test_results[i][j] += log(term_matrix[j][4])
            elif testX[i][1] == 'high':
                test_results[i][j] += log(term_matrix[j][5])
            elif testX[i][1] == 'med':
                test_results[i][j] += log(term_matrix[j][6])
            elif testX[i][1] == 'low':
                test_results[i][j] += log(term_matrix[j][7])
            #check third attribute 'doors'
            if testX[i][2] == '2':
                test_results[i][j] += log(term_matrix[j][8])
            elif testX[i][2] == '3':
                test_results[i][j] += log(term_matrix[j][9])
            elif testX[i][2] == '4':
                test_results[i][j] += log(term_matrix[j][10])
            elif testX[i][2] == '5more':
                test_results[i][j] += log(term_matrix[j][11])
            #check fourth attribute 'persons'
            if testX[i][3] == '2':
                test_results[i][j] += log(term_matrix[j][12])
            elif testX[i][3] == '4':
                test_results[i][j] += log(term_matrix[j][13])
            elif testX[i][3] == 'more':
                test_results[i][j] += log(term_matrix[j][14])
            #check fifth attribute 'lug_boot'
            if testX[i][4] == 'small':
                test_results[i][j] += log(term_matrix[j][15])
            elif testX[i][4] == 'med':
                test_results[i][j] += log(term_matrix[j][16])
            elif testX[i][4] == 'big':
                test_results[i][j] += log(term_matrix[j][17])
            #check sixth attribute 'safety'
            if testX[i][5] == 'low':
                test_results[i][j] += log(term_matrix[j][18])
            elif testX[i][5] == 'med':
                test_results[i][j] += log(term_matrix[j][19])
            elif testX[i][5] == 'high':
                test_results[i][j] += log(term_matrix[j][20])
    pred = []
    for row in test_results:
        if numpy.argmax(row) == 0:
            pred.append('unacc')
        elif numpy.argmax(row) == 1:
            pred.append('acc')
        elif numpy.argmax(row) == 2:
            pred.append('good')
        elif numpy.argmax(row) == 3:
            pred.append('vgood')    
    return pred
\end{mycode}

\begin{mycode}
# main function
if __name__ == '__main__':
    cv = 10
    t_err = 0
    for k in range(0, cv):
        trainX, trainY, testX, testY = preprocess('car.data', 0.1)
        term_matrix = term_prob_matrix(trainX, trainY)
        prior_prob = calc_prior_prob(trainY)
        prediction = predict(testX, term_matrix, prior_prob)
        err = 0
        for i in range(0, len(testY)):
            if prediction[i] != testY[i]:
                err += 1
        print(1-err/len(testY))
        t_err += 1-err/len(testY)
    print(t_err/cv)
\end{mycode}

%\section{Decision Tree in Java}
%
%Git repo from Kyle Oswald:
%\url{https://github.com/Firbydude/decision_tree}
%
%\section{Neural Network in Java}
%
%Git repo from Kyle Oswald:
%\url{https://github.com/Firbydude/neural}
%
%
%\section{Neural Network in R}
%
%The R codes written by Xueling Li are shown as follows and run on Windows. 


\end{document}
